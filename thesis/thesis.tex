% Copyright (C) 2014-2020 by Thomas Auzinger <thomas@auzinger.name>

\documentclass[draft,final]{vutinfth} % Remove option 'final' to obtain debug information.

% Load packages to allow in- and output of non-ASCII characters.
\usepackage{lmodern}        % Use an extension of the original Computer Modern font to minimize the use of bitmapped letters.
\usepackage[T1]{fontenc}    % Determines font encoding of the output. Font packages have to be included before this line.
\usepackage[utf8]{inputenc} % Determines encoding of the input. All input files have to use UTF8 encoding.

% Extended LaTeX functionality is enables by including packages with \usepackage{...}.
\usepackage{amsmath}    % Extended typesetting of mathematical expression.
\usepackage{amssymb}    % Provides a multitude of mathematical symbols.
\usepackage{mathtools}  % Further extensions of mathematical typesetting.
\usepackage{microtype}  % Small-scale typographic enhancements.
\usepackage[inline]{enumitem} % User control over the layout of lists (itemize, enumerate, description).
\usepackage{multirow}   % Allows table elements to span several rows.
\usepackage{booktabs}   % Improves the typesettings of tables.
\usepackage{subcaption} % Allows the use of subfigures and enables their referencing.
\usepackage[ruled,linesnumbered,algochapter]{algorithm2e} % Enables the writing of pseudo code.
\usepackage[usenames,dvipsnames,table]{xcolor} % Allows the definition and use of colors. This package has to be included before tikz.
\usepackage{nag}       % Issues warnings when best practices in writing LaTeX documents are violated.
\usepackage{hyperref}  % Enables cross linking in the electronic document version. This package has to be included second to last.
\usepackage[acronym,toc]{glossaries} % Enables the generation of glossaries and lists fo acronyms. This package has to be included last.
\usepackage[round]{natbib}
\usepackage{amsfonts}
\usepackage{multibib}
\usepackage{bm}

\newcites{Online}{Non-Scientific Bibliography}

% Define convenience functions to use the author name and the thesis title in the PDF document properties.
\newcommand{\authorname}{Rafael Sterzinger} % The author name without titles.
\newcommand{\thesistitle}{Balancing Extrinsic and Intrinsic Rewards in Reinforcement Learning}
%Prediction Error based Curiosity and How to Combine Extrinsic and Intrinsic Rewards
\newcommand{\thesissubtitle}{}% The title of the thesis. The English version should be used, if it exists.
\newcommand{\p}[1]{see p. #1}

% Set PDF document properties
\hypersetup{
pdfpagelayout   = TwoPageRight,           % How the document is shown in PDF viewers (optional).
linkbordercolor = {Melon},                % The color of the borders of boxes around crosslinks (optional).
pdfauthor       = {\authorname},          % The author's name in the document properties (optional).
pdftitle        = {\thesistitle},         % The document's title in the document properties (optional).
pdfsubject      = {Bachelor's thesis on \thesistitle; Written by \authorname in the context of TARO/INSO research group},         % The document's subject in the document properties (optional).
pdfkeywords     = {deep reinforcement learning, sparse rewarding environments, prediction error based curiosity, balancing extrinsic and intrinsic rewards} % The document's keywords in the document properties (optional).
}

\setpnumwidth{2.5em}        % Avoid overfull hboxes in the table of contents (see memoir manual).
\setsecnumdepth{subsection} % Enumerate subsections.

\nonzeroparskip             % Create space between paragraphs (optional).
\setlength{\parindent}{0pt} % Remove paragraph identation (optional).

\makeindex      % Use an optional index.
\makeglossaries % Use an optional glossary.
%\glstocfalse   % Remove the glossaries from the table of contents.

% Set persons with 4 arguments:
%  {title before name}{name}{title after name}{gender}
%  where both titles are optional (i.e. can be given as empty brackets {}).
\setauthor{}{\authorname}{}{male}
\setauthorextra
\setadvisor{Ao.Univ.Prof. Dipl.-Ing. Dr.techn.}{Thomas Grechenig}{}{male}

% For bachelor and master theses:
\setfirstassistant{Dipl.-Ing.}{Michael Ressmann}{}{male}

% For dissertations:
%\setfirstreviewer{Pretitle}{Forename Surname}{Posttitle}{male}
%\setsecondreviewer{Pretitle}{Forename Surname}{Posttitle}{male}

% For dissertations at the PhD School and optionally for dissertations:
%\setsecondadvisor{Pretitle}{Forename Surname}{Posttitle}{male} % Comment to remove.

% Required data.
\setregnumber{11778282}
\setdate{10}{08}{2020} % Set date with 3 arguments: {day}{month}{year}.
\settitle{\thesistitle}{\thesistitle} % Sets English and German version of the title (both can be English or German). If your title contains commas, enclose it with additional curvy brackets (i.e., {{your title}}) or define it as a macro as done with \thesistitle.
\setsubtitle{\thesissubtitle}{\thesissubtitle} % Sets English and German version of the subtitle (both can be English or German).

% Select the thesis type: bachelor / master / doctor / phd-school.
% Bachelor:
\setthesis{bachelor}
%
% Master:
%\setthesis{master}
%\setmasterdegree{dipl.} % dipl. / rer.nat. / rer.soc.oec. / master
%
% Doctor:
%\setthesis{doctor}
%\setdoctordegree{rer.soc.oec.}% rer.nat. / techn. / rer.soc.oec.
%
% Doctor at the PhD School
%\setthesis{phd-school} % Deactivate non-English title pages (see below)

% For bachelor and master:
\setcurriculum{Business Informatics }{Wirtschaftsinformatik} % Sets the English and German name of the curriculum.

% For dissertations at the PhD School:
%\setfirstreviewerdata{Affiliation, Country}
%\setsecondreviewerdata{Affiliation, Country}


\newacronym{im}{IM}{Intrinsic Motivation}
\newacronym{em}{EM}{Extrinsic Motivation}
\newacronym{rl}{RL}{Reinforcement Learning}
\newacronym{mdp}{MDP}{Markov Decision Process}
\newacronym{ml}{ML}{Machine Learning}
\newacronym{mlp}{MLP}{Multi Layer Perceptron}
\newacronym{nn}{NN}{Neural Network}
\newacronym{dp}{DP}{Dynamic Programming}
\newacronym{dl}{DL}{Deep Learning}
\newacronym{drl}{DRL}{Deep Reinforcement Learning}
\newacronym{mc}{MC}{Monte-Carlo}
\newacronym{td}{TD}{Temporal-Difference}
\newacronym{ppo}{PPO}{Proximal Policy Optimization}
\newacronym{trpo}{TRPO}{Trust Region Policy Optimization}
\newacronym{relu}{ReLU}{Rectified Linear Unit}

\begin{document}

    \frontmatter % Switches to roman numbering.
%  The structure of the thesis has to conform to the guidelines at
%  https://informatics.tuwien.ac.at/study-services

    \addtitlepage{naustrian} % German title page (not for dissertations at the PhD School).
    \addtitlepage{english} % English title page.
    \addinsotitlepage{naustrian}
    \addstatementpage

    \begin{danksagung*}

        %    Ich möchte diese Arbeit meine beiden Eltern widmen, welche mich während meines Studiums nicht nur finanzielle, sondern auch moralisch unterstützt haben. Insbesondere während der Verfassung dieser %Bachelorarbeit möchte ich mich dafür bedanken, dass ihr immer für mich da wart und mich in meinem Bestreben ermutigt habt.

        %    Ebenfalls ein großes Dankeschön möchte ich an meinen Betreuer Michael Ressmann ausprechen, welcher mir während der Planung und Durchführung der Arbeit den nötigen Freiraum gegeben hat, zeitgleich aber auch %darauf geachtet hat, dass ich mich in meinem Vorhaben nicht übernehme. Außerdem möchte ich mich bei ihm für sein gutes Feedback sowie für die rasche Benotung meiner Bachelorarbeit bedanken, welcher aufgrund %zeitlicher Beschränkungen meinerseits notwendig war.

        %    Außerdem möchte ich mich bei meinen Freunden bedanken, insbesondere bei meinen langjährigen Freunden, Stefan Zeller und Vinzenz Schicho, welche mich in diesen intensiven Wochen begleitet und mir gut zugesprochen haben.

        %Zu guter Letzt
% wöchentlichen treffen um energie zu tanken für weiteres verfassen
%TODO danke für korrektur lesen
    \end{danksagung*}

    \begin{acknowledgements*}
        % I would like to dedicate this work to my parents, who supported me not only financially but also morally during my studies. In particular while writing this Bachelor's thesis, I would like to thank both of you% for always being there for me and for encouraging me in my endeavors.

        % I would also like to express a big thank you to my supervisor Michael Ressmann, who gave me the necessary freedom during the planning and execution of this Bachelor's thesis, but at the same time also made %sure that I did not take on too much during my project. I would also like to thank him for his good feedback and for the quick grading of my thesis, which was necessary due to time constraints on my part.

        % Also, I would like to thank my friends, especially by my long-standing friends, Stefan Zeller and Vinzenz Schicho, who accompanied me during these intensive weeks and cheered me up.
        % Last but not least,
%TODO danke für korrektur lesen

    \end{acknowledgements*}

    \begin{kurzfassung}
    \end{kurzfassung}

    \begin{abstract}
    \end{abstract}

% Select the language of the thesis, e.g., english or naustrian.
    \selectlanguage{english}

% Add a table of contents (toc).
    \tableofcontents % Starred version, i.e., \tableofcontents*, removes the self-entry.

% Switch to arabic numbering and start the enumeration of chapters in the table of content.
    \mainmatter


    \chapter{Introduction}

    %TODO write introduction


    \section{Problem Description}\label{sec:problem-description}
    In \gls{rl}, algorithms mainly depend on carefully designed extrinsic reward functions which are comparable with feedback to an agent's behaviour.
    In order to teach an agent a certain desired behaviour, it has to optimize this reward function by trial and error.
    This traditional approach yielded multiple astonishing results over the last decade.
    Tasks which award rewards densely, i.e.\ almost after every performed action, are common for these results.
    In opposite to dense rewards, sparse rewards are scattered in the environment which poses the problem that they occur too rarely in order for an agent to pick up the needed skills.
    An example for environments with sparse rewards are tasks which depend heavily on exploration, e.g. \textit{Montezumas's revenge}.
    \\\\
    An approach to tackle these problems is curiosity, an organism's \gls{im} to spontaneously explore its environment.
    \gls{im} allows an agent to incrementally learn valuable skills independently of its given task by an intrinsic and therefore a more general reward function.
    In general, the concept of intrinsic and extrinsic evolves around the question on why an agent performed a certain action in a given state.
    Since \gls{im} became an important part of \gls{rl}, the following challenges were crystallized out by~\cite[\p{6}]{aubret_survey_2019}:

    \begin{itemize}
        \item \textbf{Sparse rewards:} The agent never reaches a reward signal in case of sparse rewards.
        \item \textbf{State representation:} The agent does not manage to learn a representation of its observations with independent features or meaningful distance metrics.
        \item \textbf{Building option:} The agent is unable to learn abstract high-level decisions independently from the task.
        \item \textbf{Learning a curriculum:} The agent hardly defines a curriculum among its available goals without expert knowledge.
        \label{enm:challenges}
    \end{itemize}

    Besides a general introduction to \gls{im} and its challenges in \gls{rl}, this Bachelor's thesis has its focus on knowledge acquisition through exploration.
    Knowledge acquisition is described as the agents motivation to find new knowledge about its environment, meaning that it is interested in things it can or cannot control, the function of the world, discovering new areas, or understanding the proximity.
    Concerning exploration, one approach is error prediction which is the agents difficulty to predict the state following a state-action tuple.
    This idea is heavily explored by~\cite{burda_large-scale_2018-1} and its prior work of~\cite{pathak_curiosity-driven_2017-1}, both of which are core references of this thesis.
    \\\\
    Building upon the findings of~\citeauthor{burda_large-scale_2018-1}, this Bachelor's thesis eyes on answering the question on how to optimally combine intrinsic and extrinsic rewards in order to maximize an agents score, primarily in sparse but also in dense environments.
    Furthermore, it will evaluate if a balanced reward combination tackles the noisy-TV problem, proposed by~\cite{schmidhuber_formal_2010}.
    The noisy-TV problem, also known as the white-noise problem, is an algorithms inability to handle the local stochasticity of the environment: "\textit{\ldots random noise in a 3D environment attracts the agent; it will passively watch the noise since it will not be able to predict the next state.}"\cite[\p{10}]{aubret_survey_2019}
    \\\\
    The domain of the given problem description is \gls{rl} with a focus on the model of knowledge acquisition via exploration.


    \section{Expected Results}\label{sec:expected-results}
    As aforementioned, the goal pursued by this Bachelor's thesis is to build upon the results of~\citeauthor{burda_large-scale_2018-1} and to pursue their posed question on how to optimally combine intrinsic ($r_{int}$) and extrinsic ($r_{ext}$) rewards.
    In order to do so, coefficients $\alpha$ and $\beta$ have to be selected accordingly.
    \[r=\alpha r_{int} + \beta r_{ext}\]
    From that, an optimal solution to this equation maximizes the agent's overall reward when testing.
    \\\\
    After a re-implementation of the proposed algorithms, the benchmarks are expected to at least match the mean rewards mentioned in the \textit{Additional Results} section from the underlying paper.
    In this section, the authors explored the performance of combined rewards on five different Atari games and already showed that the combination of rewards yield a higher mean reward and thus an exhaustive hyper-parameter tuning should certainly improve these results.
    However, this poses the question on how algorithms with combined rewards perform in densely rewarding environments as well.
    Therefore, it would be interesting to observe the performance in environments with dense rewards too since if there might be an improvement to see as well.
    \\\\
    In the \textit{Discussion} section,~\citeauthor{burda_large-scale_2018-1} mention the limitations of prediction error based curiosity which was earlier introduced with the so-called noisy-TV problem.
    Given this problem, the question is raised on whether or not it is possible to tackle the agent's distraction through curiosity with \gls{em}.
    This question will be pursued during the evaluation process with the posed hypothesized that it is possible to overcome an agent's distraction and the assumption that \gls{em} has a higher impact on the distraction in a densely rewarding environment than in a sparse one.
    \\\\
    With the given prospect to beat state of the art approaches to environments with sparse rewards and evaluate the amount of extrinsic reward needed to put a distracted agent back on track, it is the opportunity to pursue important constructs, which reflect the natural human propensity to learn and assimilate, in the domain of computer science that motivated this Bachelor's thesis.


    \section{Methodological Approach}\label{sec:methodological-approach}
    In order to answer the question on how to optimally combine intrinsic and extrinsic rewards, the thesis builds upon the released source code and environments from~\cite{burda_large-scale_2018-1} and~\cite{pathak_curiosity-driven_2017-1} which can be observed on these websites. \footnote{\url{https://pathak22.github.io/large-scale-curiosity}}\footnote{\url{https://pathak22.github.io/noreward-rl/}}
    Therefore, this Bachelor's thesis follows a programming approach which relies on the same tools used by the preceding authors, since it guarantees optimal reproducibility of their results.
    In an overview, these tools are the \index{Python}\textit{Python}\footnote{\url{https://www.python.org/}} programming language, the \gls{ml} platform \index{TensorFlow}\textit{TensorFlow}\footnote{\url{https://www.tensorflow.org/}}, and lastly the \index{OpenAI Gym}\textit{OpenAI Gym}\footnote{\url{https://gym.openai.com/}} which is a toolkit for developing and comparing \gls{rl} algorithms.
    \\\\
    The development progress is structured into the following five mile stones:

    \begin{enumerate}
        \item Setting-up of the environment, including the installation of the \index{CUDA}\textit{CUDA}\footnote{\url{https://developer.nvidia.com/cuda-toolkit}} toolkit and the \index{cuDNN}\textit{cuDNN}\footnote{\url{https://developer.nvidia.com/cudnn}]} library
        \item Implementing a Proximal Policy Optimization (PPO) algorithm according to~\cite{schulman_proximal_2017}, using extrinsic rewards
        \item Adding an Intrinsic Curiosity Module, published by~\cite{pathak_curiosity-driven_2017-1}, in order to allow for intrinsic rewards
        \item Tuning of the coefficients $\alpha,\beta$ and comparison to state of the art approaches
        \item Recreation of the synthetic generated noisy-TV problem, proposed by~\cite{burda_large-scale_2018-1}
    \end{enumerate}

    During this incremental process, an emphasise is laid on two out of five sparse Atari 2600 games which were categorized by~\cite{bellemare_unifying_2016} and picked by~\citeauthor{burda_large-scale_2018-1}, namely \textit{Montezumas's revenge} and \textit{Freeway}.
    \\\\
    % Eventually add VAE
    Regarding mile stone three, \textit{Random Features} and \textit{Inverse Dynamics Features} as introduced by~\citeauthor{burda_large-scale_2018-1} are primarily the focus.
    Concerning mile stone four and the comparison to state of the art approaches, chapter~\ref{sec:state_of_the_art} of this thesis allows for a current in-depth overview on methods based on knowledge acquisition via exploration.
    Additionally to the aforementioned practical part, the theoretical aspect of this Bachelor's thesis will be covered by an exhaustive literature research with an emphasize on directly comparable approaches, i.e. prediction error based curiosity~\citep{burda_large-scale_2018-1}.


    \section{Thesis Outline}\label{sec:thesis-outline}


    \glsresetall


    \chapter{Reinforcement Learning Background}\label{ch:reinforcement-learning-background}

    %TODO write introduction


    \section{Reinforcement Learning Problem}\label{sec:reinforcement-learning-problem}
    Reinforcement Learning, the science of decision making, is about learning the consequences of performed actions and using this knowledge to maximize a numeric \textit{reward signal} which should direct the search for a given objective~\citep[\p{1f}]{sutton_reinforcement_2018}.
    \gls{rl} problems are tackled by a learner or rather an \textit{agent} that performs actions in an \textit{environment} in a trial and error paradigm.
    The agent has no prior knowledge of the consequences of the possible actions but has to explore which action to take in a certain situation to maximize the amount of obtainable rewards.
    An additional difficulty is added if the reward for an action is delayed because the agent has to understand which of the executed actions actually led to the desirable reward it received far ahead in the future.
    Lastly, a proper weighting between exploration and exploitation has to be developed as the agent must exploit actions with known consequences to perform well but also explore new ones in order to improve~\citep{kaelbling_reinforcement_1996}.

    In addition to the aforementioned environment, agent, and reward signal, \gls{rl} consists of further subelements since it would be desirable e.g. not only to know the immediate reward but also to gain a measurement for all possible future rewards from a given state onward.
    For this purpose, a \textit{value function} is used.
    It acts as a long-term indicator that describes the attractiveness of a state an agent might result in after interacting with its environment.
    This measurement is interesting because there could be an action which leads to a high immediate reward but results in an undesirable state with no future rewards.
    Furthermore, the agent should pick up a \textit{policy} that consists of state-action tuples which should be refined and adjusted depending on the received rewards.
    Lastly, the agent may want to build a \textit{model} which estimates the environment's behaviour, in order to plan its actions in advance.
    This final addition divides the \gls{rl} problem spectrum into two parts: model-free and model-based approaches~\citep[6f]{sutton_reinforcement_2018}.


    In opposite to other known \gls{ml} disciplines such as \textit{supervised learning}, there is no supervisor in \gls{rl} that explicitly tells the agent which action would be correct in a given state and because of that, the agent has to come up with its own policy based on the repeated feedback it receives from the immediate reward and the following state~\citep{kaelbling_reinforcement_1996}.
    This spares the necessity of creating a training dataset with corresponding state-action tuples beforehand and thus \gls{rl} is the preferred choice in cases where an agent interacts with and influences its environment.
    Furthermore, teaching based on supervision would also mean that an agent will be limited to the labelling ability of its supervisor which is another reason why it must be able to learn from its own experiences~\citepOnline{silver_lecture_2015-2}.
    Since \gls{rl} is quite distinct from supervised learning and does not rely on labelled examples, it is sometimes misclassified as \textit{unsupervised learning} which has the purpose to find structure in unlabelled data.
    However, this is not applicable for \gls{rl} as the agent has to maximize a reward function and not to find structure in data.
    Therefore, \citet[\p{2}]{sutton_reinforcement_2018} classified \gls{rl}  as a third subfield of \gls{ml}.


    \section{Markov Decision Processes}
    In order to formalize the \gls{rl} problem, the idea and notation of an \gls{mdp} is used to notate the sequential interactions between an agent and its environment.
    At every discrete time-step, $t = 0,1,2,3, \ldots \in \mathbb{N}$, the agent observes the current state, $S_t \in \mathit{S}$, of the environment and selects an available action in a given state, $A_t \in \mathit{A}(s)$.
    Sometimes this notation is simplified to $A_t \in \mathit{A}$ assuming that the action set $\mathit{A}$ does not depend on the state~\citep[\p{48}]{sutton_reinforcement_2018}.
    Once the agent has selected an action, it obtains a scalar reward $R_{t+1} \in \mathit{R} \subset \mathbb{R}$ and depending on the action, the environment transforms into a subsequent state $S_{t+1}$~\eqref{fig:rl_problem}.
    By means of continuous interactions between the agent and the environment, a trajectory of the form $S_0,A_0,R_1,S_1,A_1,R_2,S_2,A_2,R_3,\ldots$ is created.

    \begin{figure}[h]
        \centering
        \includegraphics[width=\textwidth]{figures/rl_problem.png}
        \caption[The \gls{rl} framework formalized as an \gls{mdp}]{The \gls{rl} framework formalized as an \gls{mdp}\protect\footnotemark}
        \label{fig:rl_problem}
    \end{figure}

    \footnotetext{\cite[\p{48}]{sutton_reinforcement_2018}}

    In a finite horizon, the agent has to plan only a fixed number of time-steps ahead and the sets $\mathit{S},\mathit{A}\text{, and }\mathit{R}$ are limited~\citep[\p{47f}]{sutton_reinforcement_2018,kaelbling_reinforcement_1996}.
    Furthermore, the possibility that the values $s'\in \mathit{S}$ and $r \in \mathit{R}$ may occur at a time-step $t+1$, can be described with a probability~\eqref{eq:distribution} solely depending on the action $a \in \mathit{A}(s)$ taken in a preceding state $s \in \mathit{S}$.

    \begin{equation}
        p(s',r|s,a) = P\{S_{t+1}=s', R_{t+1}=r | S_t=s, A_t=a\}\label{eq:distribution}
    \end{equation}

    Since $p$ describes a probability, the sum over all possible combinations of $s$ and $a$ must be equal to 1.
    Moreover, the individual probabilities of $p$ describe the dynamics of an \gls{mdp}.
    Observing this notation shows that the state $s$ must contain all the necessary information and must not depend on states preceding $s$.
    If this is the case, the state is called to have the Markov property~\eqref{eq:markov_property}~\citep{francois-lavet_introduction_2018}.

    \begin{equation}
        P\{S_{t+1}|S_t,A_t\} = P\{S_{t+1}|S_1,A_1,\ldots,S_t,A_t\} \label{eq:markov_property}
    \end{equation}

    With the notation of $p$, properties of the \gls{mdp} like the state-transition probabilities~\eqref{eq:transition} or the expected value of rewards~\eqref{eq:expected_reward} with given state-action tuples can be computed.

    \begin{equation}
        p(s'|s,a) = P\{S_{t+1}=s'| S_t=s, A_t=a\} = \sum_{r \in \mathit{R}} p(s',r | s,a) \label{eq:transition}
    \end{equation}

    \begin{equation}
        r(s',s,a) = \mathbb{E}[R_{t+1} | S_t=s, A_t=a] = \sum_{r\in \mathit{R}} r \sum_{s' \in \mathit{S}} p(s',r | s,a) \label{eq:expected_reward}
    \end{equation}

    As mentioned in the~\nameref{sec:reinforcement-learning-problem}, rewards at each time-step $R_t$ are used to formalize a goal which the agent should pursue.
    This is possible due to the reward hypothesis which states that every goal one can possibly think of, can be formalized by means of a numeric reward signal and be achieved by maximizing the cumulative long-term rewards~\citep[\p{53}]{sutton_reinforcement_2018}.

    Mathematically, the overall reward is denoted as the sum over all rewards~\eqref{eq:sum_of_rewards}, with $T$ as the final time-step.

    \begin{equation}
        G_t=R_{t+1} + R_{t+2}+ \ldots + R_{T}  =\sum_{t=1}^{T} R_{t}\label{eq:sum_of_rewards}
    \end{equation}

    However, this notation is only applicable if an environment has an identifiable terminal state.
    When this state is reached, it breaks for instance a game sequence into multiple ones where each one is called an episode.
    No matter how the agent reached the final state, the environment will reset and start over with a default initial state.

    Sometimes, this separation of episodes might not be as clear and the limit of the sequence is infinite.
    This is called a \textit{continuing task}, in contrast to \textit{episodic tasks} where $T$ is finite.
    Therefore, if the agent acts in an environment with a continuing task, optimizing the equation~\eqref{eq:sum_of_rewards} might not be finite.
    This issue is solved by introducing a discounting rate $\gamma$, $0 \leq \gamma \leq 1$ which avoids infinite rewards~\citep[\p{54f}]{sutton_reinforcement_2018}.
    Given this discounting rate, the agent now tries to maximize the discounted sum of cumulative long-term rewards~\eqref{eq:discounted_reward}.

    \begin{equation}
        G_t = R_{t+1} + \gamma R_{t+2}+ \gamma^2 R_{t+3} + \ldots = R_{t+1} + \gamma G_{t+1} = \sum_{k=1}^{\infty} \gamma^k R_{t+k}\label{eq:discounted_reward}
    \end{equation}

    With $\gamma$ as newly introduced hyperparameter, the weighting of the rewards can be adjusted according to the given task.
    A low $\gamma$ means that the agent values more the immediate reward and rewards in the near future.
    Whereas, a high $\gamma$ forces the agent to consider long-term future rewards as well and thus the agent is more far-sighted.


    \section{Functions to Improve the Policy}\label{sec:functions-to-improve-the-policy}
    As mentioned in section~\ref{sec:reinforcement-learning-problem}, the policy forms an important component of the \gls{rl} framework and is usually denoted as $\pi$.
    By following a policy, the agent's probability to select a specific action at time-step $t$ is $\pi(a|s) = P\{A_t=a|S_t=s\}$ and therefore $\pi$ describes a probability distribution over actions given states~\citepOnline{silver_lecture_2015-1}.
    Additionally, the aforementioned value function component is dependent on the policy, since $\pi$ defines which action-state tuples are considered from the current state onwards.
    Formally, the state-value function for an \gls{mdp} is denoted by $v_\pi(s)$ which stands for the expected future rewards when following the policy, starting in state $s$ and "\textit{$\mathbb{E}_\pi[\cdot]$ denotes the expected value of a random variable given that the agent follows policy $\pi$}"~\citep[\p{58}]{sutton_reinforcement_2018}.

    \begin{equation}
        v_\pi(s) = \mathbb{E}_\pi[G_t|S_t = s] = \mathbb{E}_\pi \Bigg[\sum_{k=1}^{\infty} \gamma^k R_{t+k} \bigg| S_t = s \Bigg]\label{eq:value_function}
    \end{equation}

    In a similar way, the action-value function $q_\pi(s,a)$ of an \gls{mdp} can be defined.
    This calculates the expected future reward when the agent takes action $a$ in state $s$ and then follows the policy $\pi$.

    \begin{equation}
        q_\pi(s,a) = \mathbb{E}_\pi[G_t|S_t = s, A_t = a] = \mathbb{E}_\pi \Bigg[\sum_{k=1}^{\infty} \gamma^k R_{t+k} \bigg| S_t = s | A_t = a \Bigg]\label{eq:quality_function}
    \end{equation}

    The policy is improved by adjusting the value functions $v_\pi$ and $q_\pi$ through the agent's obtained experience.
    Since \gls{rl} aims to solve a given objective by maximizing rewards, it implicitly goes along with optimizing the agent's policy.
    If a policy $\pi$ achieves a higher value than a different policy $\pi'$, which also implies that $v_\pi(s) \geq v_{\pi'}(s), \forall s \in \mathit{S}$, then the two policies can be ordered partially, with $\pi \geq \pi'$.
    However, there can be multiple policies performing equally well~\citep[\p{62f}]{sutton_reinforcement_2018} and for any \gls{mdp} the following theorem holds true~\citepOnline[\p{43}]{silver_lecture_2015-1}:

    \begin{itemize}
        \item There exists an optimal policy $\pi_*$ that is better than or equal to all other policies, $\pi_* \geq \pi, \forall\pi$
        \item All optimal policies achieve the optimal [state-]value function, $v_{\pi_*}(s) = v_*(s)$
        \item All optimal policies achieve the optimal action-value function, $q_{\pi_*}(s,a) = q_*(s,a)$
    \end{itemize}

    A very simple optimal policy for a given state can be derived through the action-value function $q_*(s,a)$, simply by selecting the action with the maximum value.

    \begin{equation}
        \pi_*(a|s) =
        \begin{cases}
            1 \text{ if } a =  \underset{a \in \mathit{A}}{\text{argmax}}\ q_*(s,a),\\
            0 \text{ otherwise }
        \end{cases}
    \end{equation}

    The optimal value functions are given by the following equations:

    \begin{equation}
        \begin{aligned}[t]
            v_*(s) &= \underset{\pi}{\text{max }}v_\pi(s), \forall s \in \mathit{S}, \\
            q_*(s,a) &= \underset{\pi}{\text{max }}q_\pi(s,a), \forall s \in \mathit{S} \land \forall a \in \mathit{A}\label{eq:optimal}
        \end{aligned}
    \end{equation}


    \section{Dynamic Programming}

    The value function, introduced in section~\ref{sec:functions-to-improve-the-policy}, has a similar recursive property as the equation for the discounted expected reward~\eqref{eq:discounted_reward}.
    Therefore, the value function can be expressed in an equal manner by decomposing it into the discounted descendant states $\gamma v_\pi(s')$ and the immediate reward $r$:

    \begin{equation}
        \begin{aligned}[t]
            v_\pi(s) &= \mathbb{E}_\pi[G_t|S_t=s] = \mathbb{E}_\pi[R_{t+1} + \gamma G_{t+1}|S_t=s] \\
            &  =  \sum_{a} \pi(a|s) \sum_{s'}\sum_{r} p(s',r|s,a) \bigg[r + \gamma \mathbb{E}_\pi[G_{t+1}|S_{t+1} = s'] \bigg] \\
            &  =  \sum_{a} \pi(a|s) \sum_{s'}\sum_{r} p(s',r|s,a) \bigg[r + \gamma v_{\pi}(s') \bigg], \forall s \in \mathit{S} \\
            &  =  \mathbb{E}_\pi[R_{t+1} + \gamma v_\pi(S_{t+1})|S_t=s]
            \label{eq:bellman_equation}
        \end{aligned}
    \end{equation}

    The above mentioned equation~\eqref{eq:bellman_equation} is known as the Bellman equation for $v_\pi$~\citep[\p{59}]{sutton_reinforcement_2018} which can be directly solved due to its linearity.
    However, this calculation is only feasible for small \glspl{mdp} since its complexity is $O(n^3)$ for $n$ states~\citepOnline{silver_lecture_2015-1}.

    For larger \glspl{mdp} exist multiple iterative solutions as for example:
    \begin{itemize}
        \item \gls{dp}
        \item \gls{mc} Evaluation
        \item \gls{td} Learning
    \end{itemize}

    The general idea of \gls{dp}, which was developed by \citeauthor{bellman_theory_1954}, is to make use of the property of recursion to simplify the complexity of calculations by storing interim results.
    In that sense, it is a method that aims at solving complex problems by breaking them down into manageable subproblems.

    All of the three above mentioned approaches are mainly used for two aspects in \gls{rl}: predicting the value function $v_\pi$ (planning) and calculating the optimal value function $v_{\pi_*}$ (controlling).
    As for \gls{dp}, an iterative process is is used.
    This process is an interplay between the Bellman equation~\eqref{eq:bellman_equation}, used to calculate the value function for a given policy, and the equations mentioned in~\eqref{eq:optimal} which updates the policy~\citepOnline{silver_lecture_2015}.


    \section{Temporal-Difference Methods}

    %Write introduction for td using page 119 of sutton

    \subsection{Temporal Difference Prediction}
    Besides \gls{dp} and \gls{mc} evaluation, \gls{td} prediction is among one of the three mentioned techniques, combining the advantages of the other two for value function predictions.
    Firstly, it uses the benefit of bootstrapping offered by \gls{dp}~\citep[\p{18}]{szepesvari_algorithms_2010}.
    That is the usage of predictions of the value function as the target during the learning process.
    Secondly, \gls{td} prediction directly learns from raw experience in a similar way to \gls{mc} evaluation and thus it is model-free.
    However, it only awaits the reward of one time-step to form a target for the estimated value function $V$ instead of the every-visit \gls{mc} method which awaits the fully known return for the target before approximation~\citep[\p{120}]{sutton_reinforcement_2018}.

    \begin{equation}
        \begin{aligned}[t]
            \text{\gls{mc} method: } & V(S_t) \leftarrow V(S_t) + \alpha [G_t - V(S_t)]\\
            \text{\gls{td} method: } & V(S_t) \leftarrow V(S_t) + \alpha [R_{t+1} + \gamma V(S_{t+1})-V(S_t)]\\
        \end{aligned}\label{eq:td_versus_mc}
    \end{equation}

    The parameter $\alpha$ is a constant which defines the step-size of each adjustment to the value function.
    The two equations express the above mentioned differences certainly well:

    \begin{quote}
        "Whereas Monte Carlo methods must wait until the end of the episode to determine the increment to $V(S_t)$ (only then is $G_t$ known), TD methods need to wait only until the next time-step.
        At time $t + 1$ they immediately form a target and make a useful update using the observed reward $R_{t+1}$ and the estimate $V(S_{t+1})$."

        \hfill~\cite[\p{120}]{sutton_reinforcement_2018}
    \end{quote}

    The \gls{td} method is also known as $\text{TD}(0)$ that refers to the general case $\text{TD}(\lambda)$, a method that unifies \gls{td} and \gls{mc}.

    \subsection{SARSA (On-Policy)}\label{subsec:sarsanullon-policynull}

    SARSA builds upon the idea of \gls{td} but aims to find an optimal value function and thus solve the control problem.
    In a similar manner to \gls{td}, SARSA aims to estimate the action-value function $q_\pi(s,a)$, given a current policy $\pi$.
    Since it only focuses on one policy at a time, it is also known as an on-policy algorithm.
    The action-value function is optimized by means of the following calculation:

    \begin{equation}
        Q(S_t,A_t) \leftarrow Q(S_t,A_t) + \alpha [R_{t+1} + \gamma Q(S_{t+1},A_{t+1}) - Q(S_{t},A_{t}) ]
    \end{equation}

    At each transition between the state-action tuple, the values $S_{t},A_{t},R_{t+1},S_{t+1},A_{t+1}$ are included in the optimization process, eponymous for SARSA~\citep[\p{129}]{sutton_reinforcement_2018}.
    Under the assumption that every state-action tuple is visited infinite times and that new policies are created greedily with respect to the current action-value function, SARSA is guaranteed to converge to the optimal action-value value function and thus to the optimal policy.

    \begin{algorithm}
        \caption[SARSA for estimating $Q \approx q_*$]{SARSA for estimating $Q \approx q_*$\protect\footnotemark}
        \label{alg:sarsa}
        \SetKw{Init}{initialize}
        \SetKw{Choose}{choose}
        \SetKw{Take}{take}
        \SetKw{Observe}{observe}
        \SetKw{Update}{update}

        \KwIn{Step size $\alpha \in (0,1]$, small $\epsilon > 0$}
        \Init{$Q(s,a)$, for all $s \in \mathit{S^+}, a \in \mathit{A}(s)$, arbitrarily except that $Q(\text{terminal},\cdot)=0$}\;
        \ForAll{episodes}{
        \Init{$S$}\;
        \Choose{$A$ from $S$ using policy derived from $Q$ (e.g. $\epsilon$-greedy)}\;
        \While{$S$ is non-terminal}{
        \Take{action $A$}\;
        \Observe{$R,S'$}\;
        \Choose{$A'$ from $S'$ using policy derived from $Q$ (e.g. $\epsilon$-greedy)}\;
        \Update{$Q(S,A) \leftarrow Q(S,A) + \alpha [R + \gamma Q(S',A') - Q(S,A) ]$}\;
        \Update{$S\leftarrow S'$, $A\leftarrow A'$\;}
        }
            }



    \end{algorithm}

    \footnotetext{\citep[\p{130}]{sutton_reinforcement_2018}}
    \newpage

    \subsection{Q-Learning (Off-Policy)}\label{subsec:q-learningnulloff-policynull}

    Another control algorithm which is build upon the idea of \gls{td} is Q-learning which aims to directly estimate the optimal action-value function $q_*$.
    As denoted by \citeauthor{watkins_q-learning_1992}, the following calculation is used to update the action-value function $Q$.

    \begin{equation}
        Q(S_t,A_t) \leftarrow Q(S_t,A_t) + \alpha [R_{t+1} + \gamma \underset{a}{\text{max}} Q(S_{t+1},a) - Q(S_{t},A_{t}) ]
    \end{equation}

    As it does not optimize for the current policy but directly for the optimal one, Q-learning is considered an off-policy method and the policies must no be interchanged~\citep[\p{57}]{szepesvari_algorithms_2010}.

    \begin{algorithm}
        \caption[Q-learning for estimating $\pi \approx \pi_*$]{Q-learning for estimating $\pi \approx \pi_*$\protect\footnotemark}
        \label{alg:q_learning}
        \SetKw{Init}{initialize}
        \SetKw{Choose}{choose}
        \SetKw{Take}{take}
        \SetKw{Observe}{observe}
        \SetKw{Update}{update}

        \KwIn{Step size $\alpha \in (0,1]$, small $\epsilon > 0$}
        \Init{$Q(s,a)$, for all $s \in \mathit{S^+}, a \in \mathit{A}(s)$, arbitrarily except that $Q(\text{terminal},\cdot)=0$}\;
        \ForAll{episodes}{
        \Init{$S$}\;
        \While{$S$ is non-terminal}{
        \Choose{$A$ from $S$ using policy derived from $Q$ (e.g. $\epsilon$-greedy)}\;
        \Take{action $A$}\;
        \Observe{$R,S'$}\;
        \Update{$Q(S,A) \leftarrow Q(S,A) + \alpha [R + \gamma \underset{a}{\text{max}} Q(S',a) - Q(S,A) ]$}\;
        \Update{$S\leftarrow S'$\;}
        }
            }



    \end{algorithm}

    \footnotetext{\citep[\p{131}]{sutton_reinforcement_2018}}


    \section{Policy Gradient Methods}\label{sec:policy-gradient-methods}
    The algorithms introduced until now had all one thing in common: they were all action-value based methods.
    This means that values for given actions must be estimated in advance before a policy, based on the best achievable value in each state, can be derived.
    Thus, a policy would not even exist without these estimations~\citep[\p{321}]{sutton_reinforcement_2018}.

    In contrast, this section takes methods into account that directly learn a \textit{parameterized policy} which is not based on a previously learned value function.
    However, there are also methods which consider the best of both worlds similar to action-value based approaches, but the policy is not implicitly derived and the value function is not required for the selection of an action.
    Moreover, the value function is rather used to facilitate the learning process of the policy's parameters.
    These methods are so-called \textit{actor-critic methods}~\citepOnline{silver_lecture_2015-3} where the 'actor' stands for the learned policy and the 'critic' is a reference to the approximated value function~\citep[\p{321}]{sutton_reinforcement_2018}.

    The parameters of a policy $\pi$ are normally denoted in vector notation by $\boldsymbol{\theta} \in \mathbb{R}^{d'}$ and the policy is rewritten as $\pi(a|s,\boldsymbol{\theta})=P\{A_t=a|S_t=s,\boldsymbol{\theta}_t=\boldsymbol{\theta}\}$.
    More precisely, at time-step $t$, the probability of an action $a$ is given by the current state $s$ of the environment and the given parameters $\boldsymbol{\theta}$.
    In order to learn a policy directly, the parameters are adapted based on the derivative with respect to $\boldsymbol{\theta}$ of a numeric performance measure $J(\boldsymbol{\theta})$.
    In opposite to algorithms discussed in chapter~\ref{ch:deep-reinforcement-learning}, this derivative or rather this gradient is used to maximize the policy's performance.
    With the usage of \textit{gradient ascent} in $J$, the parameters are updated by the following equation:

    \begin{equation}
        \boldsymbol{\theta}_{t+1}=\boldsymbol{\theta}_{t} + \alpha \widehat{\nabla J(\boldsymbol{\theta_t})}\label{eq:parameter_update}
    \end{equation}

    The expectation of $\widehat{\nabla J(\boldsymbol{\theta_t})} \in R^{d'}$ is an estimation of the gradient of the performance measure $J(\boldsymbol{\theta_t})$ and $\alpha$ specifies a step-size parameter~\citepOnline{silver_lecture_2015-3}.
    Methods which use this gradient ascent to approach the parameters $\boldsymbol{\theta_t}$ for the policy $\pi$ are so-called \textit{policy gradient methods}

    In a discrete action space with few choices, state-action pairs are usually parameterized by using a scalar preference function $h(s,a,\boldsymbol{\theta})$, e.g.\ a linear combination of features, $h(s,a,\boldsymbol{\theta})=\boldsymbol{\theta}^T\boldsymbol{x}(s,a)$~\citep[\p{321}]{sutton_reinforcement_2018},~\citepOnline{silver_lecture_2015-3}.
    Moreover, policies are required to be stochastic and thus the most preferred action in a given state obtains the highest probability to be selected, e.g.\ by a softmax distribution~\eqref{eq:soft_max_policy}.
    For additional information, see also subsection~\ref{subsec:loss-functions}.

    \begin{equation}
        \pi(a|s,\boldsymbol{\theta})=\frac{e^{h(s,a,\boldsymbol{\theta})}}{\sum_{b}e^{h(s,b,\boldsymbol{\theta})}}=\frac{e^{\boldsymbol{\theta}^T\boldsymbol{x}(s,a)}}{\sum_{b}e^{\boldsymbol{\theta}^T\boldsymbol{x}(s,b)}}\label{eq:soft_max_policy}
    \end{equation}


    In summary, the utilization of policy gradient based methods gain four advantages compared to action-value based methods~\citep[\p{322f}]{sutton_reinforcement_2018}:

    \begin{enumerate}
        \item The parametrization of the policy under the usage of softmax for defining action preferences, allows for the approximation of deterministic policies.
        Action-value based methods which select actions with an $\epsilon\text{-greedy}$ approach on the other hand must always leave the option open for a random action to be selected, with probability $\epsilon$.
        Therefore, these methods cannot approach a deterministic policy.
        Even if the selection of actions with respect to action-value tuples is based on the softmax distribution as well, the policy would not approach a deterministic one.
        Rather, it would converge to the action-value tuples' true values which results in values differing the probabilities 0 and 1.
        \item In comparison to the previous advantage, it furthermore allows for arbitrary action selection which may be useful in certain types of tasks.
        For instance, playing the game of rock, paper, scissors with a deterministic policy, e.g.\ always selecting rock, the strategy would be easily exploitable.
        In this case, a stochastic, uniform random policy, which does not allow for predictions, is much better suited ~\citepOnline{silver_lecture_2015-3}.
        \item In certain situations, it might be more efficient to parameterize the policy instead of the action-value parametrization as the approximation of policies could be less complex.
        %TODO ref to image of breakout
        For instance, figuring out precisely how much score an agent can obtain in the Atari 2600 game \textit{Breakout} from one point onward when moving left might be very difficult.
        However, creating a policy that always moves left if the ball comes from a certain direction will be a lot easier to approximate~\citepOnline{silver_lecture_2015-3}.
        In these situations, policy-based methods will achieve a superior policy and converge faster\citep[\p{323}]{sutton_reinforcement_2018}.
        \item Lastly, the initial configuration of a policy's parameters enables the possibility to supply an agent with prior knowledge about an optimal policy.
        Based on this property, this is usually the main reason why policy based methods are taken into consideration.
    \end{enumerate}

    \glsresetall


    \chapter{Deep Reinforcement Learning}\label{ch:deep-reinforcement-learning}


    In the preceding chapter dealing with the~\nameref{ch:reinforcement-learning-background}, multiple solutions to the \gls{rl} problem were presented.
    However, approaches as SARSA~\eqref{subsec:sarsanullon-policynull} or Q-learning~\eqref{subsec:q-learningnulloff-policynull} only perform well if each state of $\mathit{S}$ is visited countless times in order to estimate the optimal policy $\pi$.
    In the case of video games where states are often represented as the current frame of the game, the dimension of the state-space $\mathit{S}$ is intangible which is also known as the curse of dimensionality~\citep[\p{151}]{goodfellow_deep_2016}.
    That is because each frame consists of thousands of pixels and each of these pixels can accept various values ranging from 0 to 255.
    To solve this issue, \gls{dl} techniques which emerged from \glspl{nn} are applied to \gls{rl} due to their capability of estimating very complex functions such as the value function for a high dimensional state-space.
    By combining these techniques the field of \gls{drl} emerged which enables superhuman-like problem solving skills~\citep{francois-lavet_introduction_2018}.

    Instead of presenting right away adapted traditional \gls{rl} approaches, this chapter starts with an introduction to \glspl{nn} and \gls{dl}.
    Afterwards, the chapter contains an overview on traditional \gls{drl} approaches which is finally concluded with one specific optimization method, the so-called~\gls{ppo}~\eqref{sec:proximal-policy-optimization}.
    A focus is laid on this approach because \gls{ppo} represents the algorithm which is used in the work of \citeauthor{burda_large-scale_2018-1}, the core literature of this thesis.


    \section{The Basic Architecture of Neural Networks}\label{subsec:the-basic-architecture-of-neural-networks}


    As many ideas presented in this thesis, \gls{nn} are also inspired by nature as its structure is a resemblance of the human brain.
    The name already indicates that a \gls{nn} is composed by \textit{neurons} which are connected with one another by \textit{dendrites} and \textit{axons}.
    Additional connections between axons and dendrites, called \textit{synapses}, define how important a transferred signal between them is.
    Similar to a human synapse, the strength of an artificial one is changed via a learning process which refines the weighting of the signal~\citep[\p{1f}]{aggarwal_neural_2018}.
    This idea of representation was introduced by \citeauthor{mcculloch_logical_1943} who started with a simple model which was later on adapted and improved by \citeauthor{rosenblatt_perceptron_1957}~\citep[\p{167}]{awad_deep_2015}.

    \begin{figure}[h]
        \centering
        \includegraphics[width=0.75\textwidth]{figures/neuron_model.png}
        \caption[From a biological neuron to a mathematical model: an artificial neural network in its simplest form]{From a biological neuron to a mathematical model in its simplest form: an artificial neural network\protect\footnotemark}
        \label{fig:neuron_model}
    \end{figure}

    \footnotetext{\url{https://cs231n.github.io/neural-networks-1/}}

    In its simplest form, this \gls{nn} is called a \textit{perceptron} or \textit{feed-forward network}, with one input layer and one output node, see figure~\ref{fig:neuron_model}.
    In this form, the \gls{nn} allows for a generalized variation of a linear function.
    A general perceptron is of the form $(\bar{\boldsymbol{X}},y)$, where $\bar{\boldsymbol{X}}=[x_0,x_1,\ldots,x_d]$ are the \textit{input variables} in vector notation and $y \in \mathbb{R}$  represents the \textit{observed value}.
    The observed value stands for a numerical scalar used for a regression problem, e.g. predicting the price of a house given some numerical input variables such as the amount of rooms and square meters.
    In the case of classification, this perceptron allows for binary classification where the observed value is $y$ $\in [-1,+1]$.
    For instance, this can be used to predict whether or not a credit-card transaction was fraudulent given the amount and frequency of the transaction.
    Given the input values $\bar{\boldsymbol{X}}$ and the weights $\bar{\boldsymbol{W}}=[w_0,w_1,\ldots,w_d]$, a result of a linear function is calculated by the following prediction:

    \begin{equation}
        \bar{\boldsymbol{X}}\cdot\bar{\boldsymbol{W}}=\sum_{j=0}^{d}x_j*w_j\label{eq:feed_forward_equation}
    \end{equation}

    Finally, a prediction $\hat{y}$ for $y$, based on the equation~\eqref{eq:feed_forward_equation}, is performed by activating the result making usage of an \textit{activation function}.
    In the case of classification the \textit{sign activation function} is used to map the prediction $\hat{y}$ to the previously specified range of $[-1,1]$.
    After the activation of the output value, an error between the prediction and the actual outcome can be calculated, $E(\bar{\boldsymbol{X}}) = y - \hat{y}$~\citep[\p{5ff}]{aggarwal_neural_2018}.
    If the error $E(\bar{\boldsymbol{X}})$ is non-zero, the weights must be adapted in order to perform a closer prediction for the next time.
    By optimizing the error function or \textit{loss function}, the classification error is reduced.
    The optimization process is also known as \textit{gradient descent} which updates the weights according to the negative direction of the function's gradient with respect to each input~\citep[\p{7}]{aggarwal_neural_2018} and is formally described by the following notation:

    \begin{equation}
        \bar{\boldsymbol{W}} \Leftarrow \bar{\boldsymbol{W}} + \alpha E(\bar{\boldsymbol{X}})\bar{\boldsymbol{X}}\label{eq:weight_adjusting}
    \end{equation}

    The parameter $\alpha$ is the \textit{learning rate} which defines how large the updates to the weights should be.
    A large value of $\alpha$ allows for faster convergence to a local optima with the accompanying risk of never reaching one at all.
    On the other hand, a small value steadily approaches the optima but takes longer.


    As the perceptron represents a linear function, it must not only learn the slope but also the intercept, the invariant part.
    The intercept value is also called \textit{bias} and is especially necessary in scenarios where all the features are centered around the mean and the values of the target class are not centered around the origin~\citep[\p{6}]{aggarwal_neural_2018}.
    To solve this issue, the bias variable is introduced to the perceptron as an additional neuron and an extra weight which is adjusted during the error optimization process.
    This additional neuron solely has the purpose to feed the scalar 1 to the output node and by refining the weight in between, the intercept value is learned.
    By incorporating the bias neuron, the following equation emerges:

    \begin{equation}
        \bar{\boldsymbol{X}}\cdot\bar{\boldsymbol{W}} + b=\sum_{j=0}^{d}x_j*w_j + b\label{eq:linear_function_equation}
    \end{equation}

    \subsection{Activation Functions}
    In general, activation functions play an important role in \glspl{nn} as they define the output values of neurons.
    However, they are especially significant for \glspl{mlp} as they determine the modeling power of the network.
    By using nonlinear functions, more sophisticated compositions can be created as they are not reducible to simple perceptrons~\citep[\p{13}]{aggarwal_neural_2018}.

    Apart from the previously noted sign function, other activation functions exist as well and a selection of them can be observed in figure~\ref{fig:activation_functions}.
    As mentioned in subsection~\ref{subsec:the-basic-architecture-of-neural-networks}, the sign function allows for binary classification of class labels.
    Furthermore, to enable the evaluation of the certitude of a \gls{nn}, the \textit{sigmoid activation function} is employed instead.
    Using this function allows for a mapping of the pre-activated value to a range between $[0,1]$, enabling an estimation of the network's certainty.
    Therefore, $\hat{y}$ indicates the likeliness of an input being a certain target class.
    In contrast to this, the \textit{identity activation function} is used to predict real numbers.
    By using the identity function, the \gls{nn} is equivalent to the \textit{least-squares regression} algorithm.
    An important property that all activation functions have in common is monotonicity and besides the identity function, they also "\textit{saturate at large absolute values at which increasing further does not change the activation much.}"\citep[\p{13}]{aggarwal_neural_2018}

    \begin{figure}[h]
        \centering
        \includegraphics[width=\textwidth]{figures/various_activation_functions.png}
        \caption[Various activation functions and their relation between the domain and the codomain]{Various activation functions and their relation between the domain and the codomain\protect\footnotemark}
        \label{fig:activation_functions}
    \end{figure}

    \footnotetext{\cite[\p{13}]{aggarwal_neural_2018}}

    Activation functions are applied on the pre-activated value and therefore a formal notation of $\hat{y}$ is denoted by $\hat{y}=\phi(\bar{\boldsymbol{W}}\cdot\bar{\boldsymbol{X}})$.
    Over the course of the advancement of \gls{nn}, initial activation functions such as the Sigmoid and the \textit{Tangens-Hyperbolicus} activation function have been replaced by piecewise linear functions such as the \textit{\gls{relu} activation function}.

    \begin{align*}
        &\text{Sigmoid: } &  \phi(v) & = \frac{1}{1+e^{-v}} \\
        &\text{Tangens-Hyperbolicus: } &  \phi(v) &= \frac{e^{2v}-1}{e^{2v}+1} \\
        &\text{\gls{relu}: } &  \phi(v)& = \vphantom{\frac11} \max\{v,0\}
    \end{align*}

    This is due to the fact that it facilitates the training process of \glspl{mlp} as they are less expensive to compute.

    \subsection{Loss Functions}\label{subsec:loss-functions}
    In subsection~\ref{subsec:the-basic-architecture-of-neural-networks}, a generic loss function $E(\bar{\boldsymbol{X}})$ has been introduced.
    However, depending on the target and the prediction $\hat{y}$, another loss function may be better suited as certain loss functions are more sensitive to errors for classification than for predicting real numbers~\citep[\p{15}]{aggarwal_neural_2018}.
    For instance, regression tasks modeled by \glspl{nn} with an identity activation function commonly use the \textit{squared loss} $(y - \hat{y})^2$ as an error measurement.
    In the case of classification tasks with multiple class labels, output values are normally activated employing the \textit{softmax activation function}~\citep[\p{78}]{goodfellow_deep_2016}, with $\boldsymbol{v}$ being the $n$ pre-activated output values.

    \[
        \text{Softmax: } \sigma(\boldsymbol{v})_i = \frac{e^{v_i}}{\sum_{j=0}^{n}e^{v_j}}
    \]

    This function allows for a probabilistic output which means that it maps the pre-activated values to a probability distribution.
    Since the output of the softmax function is a vector of probabilities instead of a scalar value, a different loss function is needed as well.
    Additionally, classification tasks can have binary or categorical/multiple targets which further define and differentiate which loss function will be used.

    For binary targets different combinations of activation and loss functions can be utilized to obtain the same result for prediction.
    For example, binary targets are classified by means of \textit{logistic regression} which makes usage of the identity activation function to predict $\hat{y}$.
    A loss function for this prediction is defined by the following equation:

    \begin{equation}
        L=\text{log}(1+e^{-y\cdot\hat{y}})
    \end{equation}

    A further option is to simply use the sigmoid activation function which maps the output $\hat{y}$ between $0 \text{ and } 1$.
    Under the assumption that the observed value $y$ is normalized between $-1 \text{ and } 1$, the negative logarithm of $|\frac{y}{2}-\frac{1}{2}+\hat{y}|$ is deployed as a loss function~\citep[\p{15}]{aggarwal_neural_2018}.
    This loss function provides a measurement on how likely it is that the target has correctly been classified.

    Concerning categorical targets, post-activated values $\hat{y}_0,\ldots,\hat{y}_n$ are examined one by one.
    The loss for the $i$th prediction is given by the following equation:

    \begin{equation}
        L=-\text{log}(\hat{y}_i)
    \end{equation}

    This definition is known as the \textit{cross-entropy loss} and represents the extension of logistic regression~\citep[\p{15}]{aggarwal_neural_2018}.


    \section{Multi Layer Neural Networks}
    Comparing a simple perceptron to an \gls{mlp}, the major difference is the employment of multiple layers, the so-called \textit{hidden layers}.
    Instead of a perceptron where all computations are visible, hidden layers in an \gls{mlp} hide the computations~\citep[\p{17}]{aggarwal_neural_2018}.
    However, the \gls{mlp} still is a feed-forward network and the perceptron as well as the \gls{mlp} can both be denoted as a \textit{directed acyclic graph}.
    The additional layers in an \gls{mlp} can be described as a composition of multiple functions, in opposition to a perceptron where the input is directly transmitted to the output layer.
    By composing the additional layers via chain like structures, the input is fed through multiple layers before it allows a prediction of $\hat{y}$.
    For instance, an \gls{mlp} with three layers approximates the function $f(\bar{\boldsymbol{X}})$ by chaining three functions $f^{(3)}(f^{(2)}(f^{(1)}(\bar{\boldsymbol{X}})))$ together for a prediction~\citep[\p{163f}]{goodfellow_deep_2016}.

    The amount of layers or rather the length of the chain is known as the depth of a \gls{nn}/an \gls{mlp} which also established the name \textit{deep learning} (\gls{dl}).
    As aforementioned in the introduction to chapter~\ref{ch:deep-reinforcement-learning}, only \glspl{mlp} with their ability of chaining multiple functions enable the approximation of a complex value function in a high dimensional state-space.

    \begin{figure}[h]
        \vspace{0.5cm}
        \centering
        \includegraphics[width=0.75\textwidth]{figures/xor_problem.png}
        \caption[The XOR problem with the left plot being the original space using a perceptron and the right plot being the learned space using an \gls{mlp}]{The XOR problem with the left plot being the original space using a perceptron and the right plot being the learned space using an \gls{mlp}\protect\footnotemark}
        \label{fig:xor_problem}
    \end{figure}

    \footnotetext{\cite[\p{168}]{goodfellow_deep_2016}}
    The reason why \glspl{mlp} are the preferred choice instead of a perceptron is illustrated by an example of a \gls{nn} which should estimate a model for the logical operator XOR, the "exclusive or".
    In detail, the XOR functions takes two binary values $x_0,x_1$ and returns 1 if $x_1 \neq x_2$ and 0 otherwise.
    This described behaviour is also known as the target function $f^*(\bar{\boldsymbol{X}})$ of the XOR function the \gls{nn} should approximate.
    Observing figure~\ref{fig:xor_problem}, the learned linear function of a perceptron would no be able to correctly implement the XOR function.
    Applying the model learned by the perceptron to the left, the original space, an approximated linear function cannot separate these values and thus would not be able to return the correct $y$.
    This is also known under the term of \textit{linear separability}.
    However, after transforming the original space with the help of a hidden layer and a nonlinear activation function (e.g. \gls{relu}) to the space on the right side, the XOR function can be solved as the space is now linear separable~\citep[\p{166ff}; \p{32ff}]{goodfellow_deep_2016,aggarwal_neural_2018}.
    This shows that an \gls{mlp} allows for much more complex function approximation.
    With the newly introduced capability, the learning process mentioned in equation~\ref{eq:weight_adjusting} in section~\ref{subsec:the-basic-architecture-of-neural-networks} is not applicable anymore.
    Instead, a new algorithm named \textit{backpropagation} was introduced by \citeauthor{rumelhart_learning_1986} which can be looked up either in \citet[\p{21}]{aggarwal_neural_2018} or in a more formal way in \citet[\p{197}]{goodfellow_deep_2016}.

%\subsection{(Optional)Convolutional Neural Networks}

%\subsection{(Optional)Variational Auto Encoders}


    \section{Deep Learning Approach in Reinforcement Learning}
    The advancements of \gls{dl} over the last couple of years, starting around 2010, allowed the domain of \gls{rl} to evolve as well.
    Especially through the previously mentioned combination of these two fields and the new creation of the \gls{drl} domain, many astonishing accomplishments were achieved~\citep{francois-lavet_introduction_2018}.
    A summary of four different achievements which were selected by \citet[\p{374}]{aggarwal_neural_2018} are listed in the following enumeration.

    \begin{itemize}
        \item Video games are a classical example where \gls{drl} excelled over the course of the last few years.
        Especially, games from the Atari 2600 gaming console are famous environments where \gls{drl} algorithms are applied, receiving only the raw pixels as an input.
        Based on the pixels, a state representation is formed on which the agent takes actions, makes mistakes, gains experience and finally improves its decision making.
        This procedure is equivalent to the traditional \gls{rl} approach known from chapter~\ref{ch:reinforcement-learning-background}.
        However, it is now applied to a much richer state-space and thanks to \gls{dl}, the agent is able to surpass humans in these games~\citep{mnih_playing_2013,schulman_trust_2015}.
        Furthermore, the reason why video games are so popular even in modern \gls{rl} is because they function as simulations of real life microcosms that are perfectly suitable for testing algorithms before being applied in the real world.

        \item   Another very famous breakthrough occurred in 2016, when an algorithm by the name \textit{AlphaGo}~\citep{silver_mastering_2017} defeated the top-ranked players in the board game Go.
        The reason why this event was so remarkable is because Go is a very complex game and being good at it takes a lot of human-like intuition.
        Additionally, the state space is very large when compared to other board games such as e.g. chess.
        However, it was the unconventional learning process which made AlphaGo well known as it improved itself by obtaining experience by playing against another version of itself.

        \item   \gls{drl} is also considered viable option for self-driving cars.
        Although a more common approach to this problem is the utilization of supervised learning, the usage of various car sensors for decision making allows these automated cars to have a lower error rate when compared to humans.
        \item   Another important domain where \gls{drl} plays a key factor is the creation of self-learning robots.
        Here, the difficulty relies on teaching a robot basic human-like motions such as walking.
        Fortunately, the task of teaching a robot to walk can be described in the \gls{rl} framework as the robot's goal is to reach a given position as fast as possible.
        The robot's state is a description of its available limbs and motors in a continuous space.
        Therefore, \gls{dl} techniques must be applied as well to reach good performance which does not only allow the robot to learn how to walk but also how to roll and crawl.
    \end{itemize}

    All of these accomplishments are based on different \gls{drl} approaches.
    However, an exhaustive explanation of each of these algorithms would go beyond the scope of this thesis.
    Therefore, solely an overview is provided by figure~\ref{fig:drl_taxonomy} and solely \gls{ppo} is introduced.
    For further information, the paper by \citeauthor{francois-lavet_introduction_2018} is recommended.

    \begin{figure}[h]
        \centering
        \includegraphics[width=\textwidth]{figures/drl_taxonomy.png}
        \caption[A non-exhaustive taxonomy of \gls{drl} algorithms]{A non-exhaustive taxonomy of \gls{drl} algorithms\protect\footnotemark}
        \label{fig:drl_taxonomy}
    \end{figure}

    \footnotetext{\url{https://spinningup.openai.com/en/latest/spinningup/rl_intro2.html}}


    \section{Proximal Policy Optimization}\label{sec:proximal-policy-optimization}
    \gls{ppo} belongs to the family of policy gradient~\eqref{sec:policy-gradient-methods}, model-free based methods (see Figure~\ref{fig:drl_taxonomy}) that switches between gathering data by interacting with the environment and a stochastic gradient ascent optimization process of a surrogate target function.
    In comparison to equation~\ref{eq:parameter_update}, \citeauthor{schulman_proximal_2017} denote the estimator for the gradient slightly different:

    \begin{equation}
        \hat{g}=\hat{\mathbb{E}}_t \bigg [\nabla_\theta \text{log}\pi_\theta(a_t|s_t)\hat{A}_t\bigg]\label{eq:policy_gradient_method_maximization}
    \end{equation}

    Here, $\hat{A}_t$ estimates an advantage function for a time-step $t$ where the advantage function describes
    how good an action $a$ might be, given the expected return if the policy $\pi$ is followed directly.
    In general, it is denoted by $A^\pi(s,a)=Q^\pi(s,a)-V^\pi(s)$ for a given policy, according to \citeauthor{francois-lavet_introduction_2018}.
    Furthermore, $\hat{\mathbb{E}}_t[\cdot]$ is the empirical mean over some batch samples of interactions with the environment.

    In software, the estimation of the gradient $\hat{g}$ is obtained by creating a similar objective function that acts as the estimator.
    This is done by differentiating the following loss function:

    \begin{equation}
        L^{PG}(\theta)=\hat{\mathbb{E}}_t\bigg [\text{log}\pi(a_t|s_t)\hat{A}_t\bigg]\label{eq:no_clip_no_penalty}
    \end{equation}

    Additionally, \gls{ppo} is based on an idea named \gls{trpo} where objective function~\ref{eq:policy_gradient_method_maximization} is replaced by a surrogate objective~\eqref{eq:trpo_optimization} with an additional constraint~\eqref{eq:trpo_constraint} on the update size of the policy~\citep{schulman_trust_2015}.

    \begin{equation}
        \hat{\mathbb{E}}_t\bigg [\frac{\pi_\theta(a_t|s_t)}{\pi_{\theta_{\text{old}}}(a_t|s_t)}\hat{A}_t\bigg]\label{eq:trpo_optimization}
    \end{equation}
    \begin{equation}
        \hat{\mathbb{E}}_t\bigg [\text{KL}[\pi_{\theta_{\text{old}}}(\cdot|s_t),\pi_{\theta}(\cdot|s_t)]\bigg]\leq\delta\label{eq:trpo_constraint}
    \end{equation}

    Instead of the constraint~\ref{eq:trpo_constraint}, a penalty is sometimes integrated with a corresponding weighting as well, combining the equations~\ref{eq:trpo_optimization} and~\ref{eq:trpo_constraint}.
    Furthermore, KL in equation~\ref{eq:trpo_constraint} stands for the Kullback-Leibler divergence.

    With the ratio $r_t(\theta)=\frac{\pi_\theta(a_t|s_t)}{\pi_{\theta_{\text{old}}}(a_t|s_t)}\hat{A}_t$ the main objective for \gls{ppo} can now be constructed:

    \begin{equation}
        L^{CLIP}(\theta)=\hat{\mathbb{E}}_t\bigg[\text{min}(r_t(\theta)\hat{A}_t,\text{clip}(r_t(\theta),1-\epsilon,1+\epsilon)\hat{A}_t)\bigg]\label{eq:ppo_clip}
    \end{equation}

    The main objective of the \gls{ppo} differs mainly from the surrogate objective of \gls{trpo}~\eqref{eq:trpo_optimization} as it does not allow for excessively large changes to the policy.
    This is done by clipping the initial objective, if the probability ratio $r_t$ moves too far away from 1 and whether or not the advantage $\hat{A}_t$ is negative or positive.
    Thus, a restriction is placed upon the incentive to move outside of $[1-\epsilon,1+\epsilon]$, with $\epsilon$ being a hyperparameter.
    Moreover, the minimum of the equation~\ref{eq:ppo_clip} functions as a lower bound for the objective~\ref{eq:trpo_optimization}.

    Referring to the results of \citeauthor{francois-lavet_introduction_2018} where the authors tried different surrogate objectives on seven environments with three different seeds, clipping the objective~\eqref{eq:ppo_clip} with an $\epsilon=0.2$ resulted in the best normalized, averaged score of 0.82.
    By using different $\epsilon \in [0.1,0.3]$ the score varies between $[0.70,0.82]$ whereas KL-penalized objectives achieved scores between $[0.62,0.72]$ and a non clipped and penalized objective~\eqref{eq:no_clip_no_penalty} performed the worst with an average score of $-0.39$~\citep[see Table 1]{francois-lavet_introduction_2018}.

    When comparing \gls{ppo} algorithms to similar approaches such as \gls{trpo}, \citeauthor{francois-lavet_introduction_2018} list the following advantages:

    \begin{quote}
        "[T]hey are much simpler to implement, more general and have better sample complexity (empirically). [Furthermore,] \ldots PPO outperforms other online policy gradient methods, and overall strikes a favorable balance between sample complexity, simplicity, and wall-time."

        \hfill ~\citep[\p{1}]{francois-lavet_introduction_2018}
    \end{quote}

    In \gls{drl} applications where \gls{nn} share parameters between a value function and a policy, they both must be combined to form one uniform loss function.
    Additionally, to construct the objective which \gls{ppo} tries to maximize, a bonus in the form of an entropy is added to ensure enough exploration, resulting in a final target that is given by the following equation:

    \begin{equation}
        L_t^{CLIP+VF+S}(\theta) = \hat{\mathbb{E}}_t\bigg[L_t^{CLIP}(\theta) - c_1L_t^{VF}(\theta) + c_2S[\pi_\theta](s_t)\bigg]\label{eq:ppo_objective}
    \end{equation}

    In this equation, $S$ is the bonus in form of an entropy, $L_t^{VF}$ denotes the squared-error loss $(V_\theta(s_t)-V_t^{\text{targ}})^2$ and $c_1,c_2$ are coefficients which control the influence of these two terms.
    Implementing and optimizing the objective~\ref{eq:ppo_objective} is carried out by following the policy and collecting samples of interactions with the environment for $T$ time-steps.
    The gathered data is then used to apply an update to~\ref{eq:ppo_objective}.
    Additionally, the advantage function needed for objective~\ref{eq:ppo_clip} is approximated by an estimator which only considers $T$ time-steps with an index $t\in [0,T]$ as well and is denoted as follows:

    \begin{equation}
        \hat{A}_t=-V(s_t)+r_t+\gamma r_{t+1} + \dots + \gamma^{T-t+1}r_{T-1} + \gamma^{T-t}V(s_T)\label{eq:advantage_estimator}
    \end{equation}

    The \gls{ppo} algorithm shown below~\eqref{alg:ppo} puts theory into practice where in each iteration, $N$ workers gather data for $T$ time-steps, calculate the loss with these $NT$ time-steps and apply the just calculated loss in minibatches for $K$ epochs~\citep{francois-lavet_introduction_2018}.
    \newpage
    \begin{algorithm}
        \caption[\gls{ppo}, Actor-Critic Style]{\gls{ppo}, Actor-Critic Style\protect\footnotemark}
        \label{alg:ppo}
        \SetKw{Run}{run}
        \SetKw{Comp}{compute}
        \SetKw{Opt}{optimize}
        \SetKw{Update}{update}

        \For{
        $iteration=1,2,\dots$}{
        \For{$actor=1,2,\dots,N$}{
        \Run{policy $\pi_{\theta_\text{old}}$ in environment for $T$ time-steps}\;
        \Comp{advantage estimates $\hat{A}_1,\hat{A}_2,\dots,\hat{A}_T$}\;
        }
            \Opt{surrogate $L$ wrt $\theta$, with $K$ epochs and minibatch size $M \leq NT$}\;
            \Update{$\theta_\text{old}\leftarrow \theta$}\;
        }



    \end{algorithm}

    \footnotetext{\citep[\p{5}]{schulman_proximal_2017}}

    \glsresetall


    \chapter{Intrinsic Motivation in Reinforcement Learning}\label{ch:intrinsic-motivation-in-reinforcement-learning}
    The idea of including \gls{im} into the \gls{rl} framework has been around since the early 1990 as in realistic environments extrinsic rewards occur rather rare and thus quick progress by an agent cannot be expected.
    With its roots deeply involved in psychology, \gls{im} was initially used due to the ability of discovering surprising or novel patterns, enabling an agent to be creative in the sense that it learns non-trivial, new behaviour~\citep{schmidhuber_formal_2010}.
    This phenomenon is based on a discovery from nature as animals were observed to engage in curiosity-driven, exploratory, and playful behaviour despite the fact that they would not obtain a reward or a reinforcement~\citep{white_motivation_1959}.

    This chapter is started by introducing the psychological aspects of \gls{im} as well as motivation in general and is continued with the role of \gls{im} in the \gls{rl} framework.
    Afterwards, current challenges of \gls{rl} which are tackled by \gls{im} are pointed out (see Listing~\ref{enm:challenges}), and the embedding of \gls{im} in \gls{drl} is presented.
    Finally, this chapter is concluded with the classification of \gls{im} in \gls{rl}: \textit{knowledge acquisition} and \textit{skill learning} where a subdomain of knowledge acquisition is more deeply explored in chapter~\ref{sec:state_of_the_art} as it represents the domain on which the implementation~\eqref{ch:implementation} is based on.


    \section{Natural Human Motivation}

    In a digression to psychology, \gls{im} is described as the natural propensity of humans to gather knowledge and assimilate.
    Furthermore, activities which are intrinsically motivated are usually carried out because they are enjoyable and interesting.
    Therefore, the reward in \gls{im} relies in the activity itself.
    In opposite to that, \gls{em} reflects external control sources or true self-regulation which may achieve separable outcomes, e.g.\ food or money.
    Motivation in general is very unique for different people~\citep{ryan_intrinsic_2000}.
    Not only can it vary in the amount or rather the \textit{level of motivation} a person can have, but the \textit{orientation of motivation} can be different as well.
    The question on the orientation of a person's motivation is very significant since e.g.\ pupils may be very motivated to excel in a subject out of curiosity (\gls{im}) or rather out of the possibility of achieving good grades (\gls{em}) which might yield further positive benefits for them or avoid sanctions.

    Moreover, research pointed out that the quality of performance and experience is dependant on the kind of motivation propelling behaviour.
    For instance, it is shown that \gls{im} allows for an improved learning experience and more creativity.
    On the other side, \gls{em} motivated actions might be executed with disinterest, resistance or optionally with an inner acceptance as the action poses an identifiable value.
    Continuing with the previously made example on the pupils' motivation, for instance, it is the teachers job to promote inherently uninteresting actions by means of extrinsic motivation with a focus on volitional and active forms instead of controlling and passive ones.
    By doing this, the pupils are more likely to identify the action's value such that they internalize the behaviour, allowing for an improved learning experience.

    Finally, \citeauthor{ryan_intrinsic_2000} add a third type of motivation to the taxonomy of natural human motivation, the so-called \textit{amotivation} which describes activities that are perceived as irrelevant.
    Amotivation has different origins such as a feeling of incompetence to carry out a task, an action not yielding any value, or any desired outcome.
    Therefore, the taxonomy of human motivation consists of three branches: amotivation, extrinsic motivation, and intrinsic motivation, ranging "\textit{from amotivation or unwillingness, to passive compliance, to active personal commitment. With increasing internalization (and its associated sense of personal commitment) come[s] greater persistence, more positive self-perceptions, and better quality of engagement.}"~\citep[\p{60f}]{ryan_intrinsic_2000}


    \section[The Role of IM in RL]{The Role of \gls{im} in \gls{rl}}

    With the help of \gls{im}, existing challenges of \gls{rl} such as abstracting actions or the exploration of the environment should be addressed which were not able to be solved by the promising capabilities of \gls{nn}.
    As mentioned in the~\nameref{sec:problem-description}, in traditional \gls{rl} approaches, the agent is unable to explore the environment in settings where rewards are scattered.
    Furthermore, even in dense environments, the agent's learned behaviour is rather not reusable in similar or completely different tasks.
    This is due to the fact that an agent finds it very challenging to generalize obtained skills and thus is incapable of high-level decision making.
    For instance, \citeauthor{todorov_mujoco_2012} indicate that such a high-level abstraction could be the trajectory of a robot to open a door which consists of low-level actions or rather movements in four directions.

    As earlier described, the idea for \gls{im} in \gls{rl} is derived from nature, especially by the animals' urge for exploration.
    However, a probably better fitting example would be a baby's task to solve existential problems as it best reflects an \gls{rl}-agent in an unknown environment.
    For instance, in order for the baby to avoid hunger or thirst, it has to learn the consequences of its interactions.
    Although, immediate needs might be satisfied for some time, the baby still continues to explore by conducting different experiments.
    Questions on certain movements of, e.g.\ eyes, fingers, or the tongue and their expected sensory feedback are continually answered.
    Ultimately, this facilitates the baby's ability of prediction which allows for easier planning of actions.
    Furthermore, the baby constantly strives to discover new effects, as it eventually gets bored by the things it already grasps, allowing it to learn very complex behaviour by building on previous gathered knowledge.
    By following this simple algorithm of maximizing the internal joy of discovering or creating new patterns, eventually the baby might become a computer scientist~\citep{schmidhuber_formal_2010}.
    Concerning \gls{rl}, this process allows an agent to autonomously gather new skills and knowledge which then facilitates the mastering of a novel task.

    Besides the improved exploration, \gls{im} yields other benefits as well.
    For instance, an agent can learn skills incrementally and independently while not being dependant on its given main task.
    Furthermore, the agent can decide on which skill might be adequate for a certain task and thus improve them selectively and additionally it might as well come up with meaningful state representation.
    This is all possible due to the broader intrinsic reward function and the fact that this spares the necessity of an additional expert or supervisor.
    Therefore, making learning more flexible and \gls{rl} generalize more across different environments~\citep{aubret_survey_2019}.

    Additionally to the classification of subfields in \gls{ml} in section~\ref{sec:reinforcement-learning-problem}, \citeauthor{aubret_survey_2019} add a fourth field to the spectrum of learning types, differentiating between reinforced and intrinsically motivated learning (see Table~\ref{tab:type_learning}).
    Therefore, a distinction is drawn depending on whether or not expert supervision is involved that gives feedback.

    \begin{table}
        \centering
        \begin{tabular}{|l|l|l|}
            \hline
            & With \textit{feedback} & Without \textit{feedback} \\
            \hline
            Active  & Reinforcement          & Intrinsic motivation      \\
            Passive & Supervised             & Unsupervised              \\
            \hline
        \end{tabular}
        \caption[A taxonomy on the different types of learning]{A taxonomy on the different types of learning\protect\footnotemark}
        \label{tab:type_learning}
    \end{table}

    \footnotetext{\citep[\p{4}]{aubret_survey_2019}}


    \section{Challenges}

    \subsection{Sparse Rewards}

    As noticed in section~\ref{sec:problem-description}, traditional \gls{rl} algorithms perform best in environments which reward the agent densely.
    By using naive exploration approaches, e.g. $\epsilon\text{-greedy}$, good results can be accomplished with an additional option, for instance, to add a \textit{Gaussian noise} to the action selection process which further improves exploration~\citep{lillicrap_continuous_2019}.
    However, even these improved exploration methods are incapable to achieve a good performance in sparsely rewarding settings such as \textit{Montezuma's Revenge} which is a typical benchmark for such environments.
    In order for the agent to receive a reward in this game, it has to collect different items such as keys and use them to open doors.
    These event occur rather rarely and thus the agent does not receive an immediate feedback concerning the impact of its performed action (e.g.\ moving in a certain direction).
    Therefore, it is not uncommon that an agent never reaches such a reward, preventing the creation of a good policy~\citep{aubret_survey_2019}.

    This issue can be tackled by shaping a task specific, intermediary function which rewards densely.
    Hence, additional rewards are created which should point the agent in the right direction.
    Although this approach might work out well, this is usually not the case as in order to craft this function, very high expert knowledge is needed to carefully design this rewarding system.
    Yet, these function commonly yield many side effects resulting in unexpected errors as the agent might pick up an undesired behaviour by exploiting the rewarding system.

    \subsection{Building a Good State Representation}\label{subsec:building-a-good-state-representation}

    According to \citeauthor{bohmer_autonomous_2015}, a good state representation allows for good generalization, is low-dimensional, should represent values of a policy truthfully, and be markovian.
    By employing a good representation, the agent's learning process can be considerably accelerated.
    For instance, considering a navigation task, is makes a significant difference whether an agent has to find out its location and the location of its target by accessing non-linear transformed raw pixels or by obtaining the positions directly creating solely the need to check the distance between itself and the target.

    Normally, a state representation can be build using expert knowledge of a specific domain.
    However, not task-specific, hence generic, priors can be used as well with the benefit that these priors can be learned by representation-learning algorithms, creating disentangled, minimal feature spaces~\cite{bengio_representation_2014}.
    In traditional \gls{rl}, the problem of learning a good state representation is exacerbated as in order for the agent to improve, it depends on optimizing the reward signal via backpropagation, thus depending on receiving rewards densely.
    With the additional possible obstacle of noise in the raw state representation, the agent will learn nothing from the interactions it made in a sparse rewarding environment, although they might be rich in information.
    Moreover, even if the agent learns a state representation based on a reward signal, the representation will not generalize across different tasks.
    Hence, \gls{im} is a key component as this learned state representation is independent from the actual task and therefore generalizable~\citep{aubret_survey_2019}.

    \subsection{Temporal Abstraction of Actions}

    In temporal abstraction of actions, a sequence of low-level actions is summarized with a so-called \textit{intra-option policy} that is referenced to a high-level action.
    Therefore, a high-level action or a so-called option consists of multiple low-level actions or rather other options.
    By choosing an option in a certain state, the intra-option policy defines which low-level actions must be carried out in each subsequent state until the high-level action is completely carried out.
    Furthermore, the sequence length of these actions is usually fixed~\citep{aubret_survey_2019}.
    By utilizing these abstractions, the learning process can be sped up significantly.
    An additional consequence of allowing these abstractions is that the problem of delayed rewards, described in section~\ref{sec:reinforcement-learning-problem}, is tackled as the option can be directly credited.
    \citeauthor{aubret_survey_2019} describe this with the following example:

    \begin{quote}
        "[L]et us assume that a robot is trying to reach a cake on a table which is far from the robot.
        If the robot has an option \texttt{get to the table} and follows it, the robot will then only have to take the cake to be rewarded.
        Then it will be easy to associate the acquisition of the cake (the reward) to the option \texttt{get to the table}.
        In contrast, if the robot has to learn to handle each of its joints (low-level or primitives actions), it will be difficult to determine which action is responsible of the acquisition of the cake, among all executed actions."

        \hfill~\cite[\p{5f}]{aubret_survey_2019}
    \end{quote}

    Another benefit of temporal abstractions of actions is that they allow for better exploration.
    Similarly to the previously made example, an exploration task can easily be combined to an option enabling it to immediately receive a reward and therefore "making a sparse reward function dense".
    In order to obtain such an intra-option policy, there are two options: define it manually using expert knowledge or learn it using a reward function.
    However, the latter has the problem that the learned options are not generalizable across multiple task and hence are useless for exploration tasks~\citep{aubret_survey_2019}.
    Similar to the problem of~\nameref{subsec:building-a-good-state-representation}, by using \gls{im} this challenge can be tackled.

    \subsection{Building a Curriculum}
    The challenge of building a curriculum is concerned with structuring tasks in a multi-task \gls{rl} scenario, a scenario where an agent has to solve multiple tasks at once.
    In such scenarios, a so-called curriculum is employed that acts as a schedule defining the order in which different tasks should be solved.
    This idea stems from the observation that incrementally building upon gathered knowledge from similar but less complex tasks facilitates mastering more difficult ones~\citep{aubret_survey_2019}.

    For instance, a general task for an agent could be that it has to store objects such as cubes in boxes.
    However, this task can be divided in subtasks such as grabbing the cube and then putting it into a box.
    By splitting up this task into the two subtasks, the agent can take advantage of the prior learnt skill of grabbing a cube for the task of storing it in a box.
    Without this differentiation, the agent might be unable to ever fulfill the general task as it would took the agent a long list of subsequent actions or rather joint movements to succeed.

    Usually, standard methods for decomposing a general task mostly rely on export knowledge which unfortunately always goes along with the inability to scale and generalize well and thus \gls{im} finds its application to solve this issue.
    Not only does this speed up the learning process but also facilitate exploration.


    \section{Embedding}


    \section{Classification}
    4.2 Classification of the use of IM in RLOudeyer and Kaplan (2008) already proposed a classifica-tion of the different IMs where the two major models are ei-ther knowledge-based or competence-based. The first oneconsists of a comparison between agent’s predictions andreality, and the second one refers to the performance onself-generated goals. We propose a slightly different clas-sification to include skill abstraction and highlight skill ac-quisition. Our classification emphasizes two major kindsof IM in RL and is summarized in the Table 2.Knowledge acquisition: With this motivation, the agentstrives to find new knowledge about its environment.This knowledge can concern what it can/cannot con-trol, the functioning of the world, discovering newareas or understanding the sense of proximity. It isvery close to the knowledge-based classification ofOudeyer and Kaplan (2008). We will see that: 1- itcan improveexplorationin sparse rewards environ-ments, e.g. by computing an intrinsic reward basedon the novelty of the states or the information gain; 2-it can push the agent to maximize itsempowermentby rewarding the agent if it is heading towards areaswhere it controls its environment; 3- it can help theagent to learn a relevantstate representation.Skill learning: We define skill learning as the agent’sability to construct task-independent and reusableskills in an efficient way. There are two core com-ponents taking advantage of this motivation: one isabout the ability of an agent to learn arepresentationof diverse skillsin order to achieve them, the otherone is about wisely choosing the skills to learn withacurriculum. Thereby, unlike previous classifica-tions, we differentiate the motivation which builds theabstract meaning of a skill and the motivation whichchooses the skill.In figure 4, we summarize the relations between differentchallenges. For each relation, we refer to the appropriatesection in which we discuss this relation.Apart from our classification, some social motivations re-ward inequity or peace[Perolatet al., 2017; Hugheset al.,2018]. They deviate from the standard definition of an IMand are very specific to cooperative games. In this case, thereward is independent from the human, but still depends onthe feedback of another agent. Therefore, we will not detailthese works.In the next two sections, we review the state-of-the-art byfollowing the classification proposed in Table 2.

    % Please add the following required packages to your document preamble:
% \usepackage{multirow}
\begin{table}[]
\begin{tabular}{|l|ll}
\cline{1-1} \cline{3-3}
\multicolumn{1}{|c|}{\textbf{Knowledge Acquisition}}                                                                                                       & \multicolumn{1}{l|}{} & \multicolumn{1}{c|}{\textbf{Skill Learning}}                                                                                                                                   \\ \cline{1-1} \cline{3-3}
\textbf{Exploration}                                                                                                                                       & \multicolumn{1}{l|}{} & \multicolumn{1}{l|}{\textbf{Skill Abstraction}}                                                                                                                                \\ \cline{1-1} \cline{3-3}
\multirow{4}{*}{\begin{tabular}[c]{@{}l@{}}Prediction error\\ State novelty\\ Novelty as discrepancy towards other states\\ Information gain\end{tabular}} & \multicolumn{1}{l|}{} & \multicolumn{1}{l|}{\multirow{2}{*}{\begin{tabular}[c]{@{}l@{}}Building the goal space from the state space\\ Mutual information between goals and trajectories\end{tabular}}} \\
                                                                                                                                                           & \multicolumn{1}{l|}{} & \multicolumn{1}{l|}{}                                                                                                                                                          \\ \cline{3-3}
                                                                                                                                                           & \multicolumn{1}{l|}{} & \multicolumn{1}{l|}{\textbf{Curriculum Learning}}                                                                                                                              \\ \cline{3-3}
                                                                                                                                                           & \multicolumn{1}{l|}{} & \multicolumn{1}{l|}{\multirow{3}{*}{\begin{tabular}[c]{@{}l@{}}Goal sampling\\ Multi-armed bandit\\ Adversarial training\end{tabular}}}                                        \\ \cline{1-1}
\textbf{Empowerment}                                                                                                                                       & \multicolumn{1}{l|}{} & \multicolumn{1}{l|}{}                                                                                                                                                          \\ \cline{1-1}
\textbf{Learning a Relevant State Representation}                                                                                                          & \multicolumn{1}{l|}{} & \multicolumn{1}{l|}{}                                                                                                                                                          \\ \cline{1-1} \cline{3-3}
\multirow{2}{*}{\begin{tabular}[c]{@{}l@{}}State space as a measure of distance\\ One feature for one object of interaction\end{tabular}}                  &                       &                                                                                                                                                                                \\
                                                                                                                                                           &                       &                                                                                                                                                                                \\ \cline{1-1}
\end{tabular}
    \caption[Classification of the use of \glspl{im} in \gls{drl}]{Classification of the use of \glspl{im} in \gls{drl}\protect\footnotemark}
    \label{tab:clasification_im_drl}
\end{table}


    \footnotetext{\citep[\p{9}]{aubret_survey_2019}}

    \chapter{Exploration}\label{sec:state_of_the_art}


    \section{Prediction Error}
    \cite[\p{7}]{schmidhuber_formal_2010}


    \section{State Novelty}


    \section{Novelty as Discrepancy towards other States}


    \section{Information Gain}



    \glsresetall


    \chapter{Evalutation}

%TODO 9. Benchmarking Deep RL
    \citep{francois-lavet_introduction_2018}

    \glsresetall


    \chapter{Implementation}\label{ch:implementation}
    \citep{burda_large-scale_2018-1}


    \section{Proximal Policy Optimization}


    \section{Intrinsic Curiosity Module}
    \citep{pathak_curiosity-driven_2017-1}


    \section{Hyper-Parameter Tuning}
    %TODO Gridsearch


    \section{Noisy-TV Problem}
    \citep{schmidhuber_formal_2010}

    \glsresetall


    \chapter{Results}


    \glsresetall


    \chapter{Discussion}

    \backmatter
% Use an optional list of figures.
    \listoffigures % Starred version, i.e., \listoffigures*, removes the toc entry.

% Use an optional list of tables.
    \cleardoublepage % Start list of tables on the next empty right hand page.
    \listoftables % Starred version, i.e., \listoftables*, removes the toc entry.

% Use an optional list of alogrithms.
    \listofalgorithms
    \addcontentsline{toc}{chapter}{List of Algorithms}

% Add an index.
    \printindex

% Add a glossary.
    % \printglossaries

% Add a bibliography.
    \bibliographystyle{abbrvnat}
    \bibliography{core}

    \bibliographystyleOnline{abbrvnat}
    \bibliographyOnline{online}
\end{document}
